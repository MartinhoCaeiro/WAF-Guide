\documentclass[a4paper]{article}
% Pacotes necessários
\usepackage[portuguese]{babel}
\usepackage[backend=biber, style=apa, citestyle=apa, language=portuguese]{biblatex}
\usepackage{csquotes}
\addbibresource{Recursos/referencias.bib}

\usepackage{amsmath}
\usepackage{graphicx}
\usepackage{subcaption}
\usepackage{setspace}
\usepackage{siunitx} % Required for alignment
\sisetup{
  round-mode          = places, % Rounds numbers
  round-precision     = 2, % to 2 places
}
\usepackage{enumerate}
\usepackage{enumitem}
\usepackage{amsmath}
\usepackage{karnaugh-map}
\usepackage[section]{placeins}
\usepackage{geometry}
\usepackage{amssymb}
\usepackage{titling}
\usepackage[T1]{fontenc}
\usepackage{float}
\usepackage[hidelinks]{hyperref}
\usepackage{xcolor}
\usepackage{indentfirst}
\usepackage{array}
\usepackage{wrapfig} % Coloca isto no preâmbulo
\usepackage{soul}
\usepackage{afterpage}
\usepackage[toc,page]{appendix}
\newcolumntype{P}[1]{>{\centering\arraybackslash}p{#1}}
\onehalfspacing


% Comando para criar uma página vazia
\newcommand\myemptypage{
    \null
    \thispagestyle{empty}
    \addtocounter{page}{-1}
    \newpage
}

% Página de título principal
\newcommand{\firsttitlepage}{
    \begin{titlepage}
        \centering
        \vspace*{1cm}
        
        % Logos superior
        \begin{figure}[h!]
            \centering
            \includegraphics[width=6cm]{Recursos/Logos/LOGO_IPB} % Substitua pelo caminho da imagem
            \vspace{0.5cm}
        \end{figure}

        % Informações da instituição
        \large\textbf{INSTITUTO POLITÉCNICO DE BEJA} \\
        \large\textbf{Escola Superior de Tecnologia e Gestão} \\
        \large\textbf{Licenciatura em Engenharia Informática} \\
        \large\textbf{Administração de Sistemas} \\
        
        \vspace{2cm}
        
        % Título do projeto
        {\Huge \textbf{Laboratório 2}} \\
        
        \vspace{1.5cm}
        
        % Autores
    
        \large Martinho José Novo Caeiro - 23917 \\
        
        \vfill
        
        % Logo inferior
        \begin{figure}[h!]
            \centering
            \includegraphics[width=6cm]{Recursos/Logos/IPBejaESTIG.jpg} % Substitua pelo caminho da imagem
        \end{figure}
        
        \vspace{1cm}
        
        % Local e data
        {\large Beja, março de 2025}
    \end{titlepage}
}

\newcommand{\secondtitlepage}{
    \begin{titlepage}
        \centering
        \vspace*{1cm}
        
        % Informações da instituição
        \large\textbf{INSTITUTO POLITÉCNICO DE BEJA} \\
        \large\textbf{Escola Superior de Tecnologia e Gestão} \\
        \large\textbf{Licenciatura em Engenharia Informática} \\
        \large\textbf{Administração de Sistemas} \\
        
        \vspace{2cm}
        
        % Título do projeto
        {\Huge \textbf{Laboratório 2}} \\
        
        \vspace{1.5cm}
        
        % Autores
        \large Martinho José Novo Caeiro - 23917 \\

        \vspace{2cm}

        % Orientador
        \large Orientador: Professor Armando Ventura \\
        
        \vfill
        
        % Local e data
        {\large Beja, março de 2025}
    \end{titlepage}
}

\begin{document}


\pagenumbering{gobble} % Oculta numeração da página

% Primeira página de título
\firsttitlepage

\secondtitlepage


% Abstract
\section*{\LARGE\textbf{\textit{Resumo}}}

Neste relatório será abordado o processo de instalação da máquina
Servidor com o uso do AlmaLinux, os pacotes instalados e a configuração de
DNS, Apache Webserver e Virtual Hosts para a realização do Laboratório 2.
Este relatório foi realizado no âmbito da Unidade Curricular de
Administração de Sistemas (\cite{pagas}).




\vspace{1cm}
% Keywords
\textbf{Keywords:} almalinux, putty, dns, apache, virtual hosts
\newpage
%--------------------------------------------------------------------------------------------------------------------------------------

\section*{\LARGE\textbf{\textit{Abstract}}}

In this report, we will address the installation process of the Server machine using
AlmaLinux, the installed packages and the configuration of DNS, Apache webserver and
Virtual Hosts for the completion of Laboratory 2.


\vspace{1cm}
% Keywords
\textbf{Keywords:} almalinux, putty, dns, apache, virtual hosts
\renewcommand{\contentsname}{Índice}       % Título do sumário
\renewcommand{\listfigurename}{Índice de Figuras} % Título da lista de figuras

% Início do conteúdo do relatório
\newpage
\doublespacing
\tableofcontents
\listoffigures
\doublespacing

\newpage
\pagenumbering{arabic}

\section{Introdução}\label{intro}
Para a realização do laboratório é necessária a configuração de uma máquina AlmaLinux Servidor.
Em seguida são configurados os serviços DNS, Apache webserver e Virtual Hosts.
Para a configuração de todos estes serviços foi utilizado o Putty (\cite{putty})
que permitiu inserir comandos mais rapidamente.
%---------------------------------------------------------------------------------------------------------------------------
\newpage
\section{Introdução Teórica}\label{obj}
\subsection{DNS}
O DNS (Domain Name System) é um sistema hierárquico e distribuído que permite a resolução de nomes de domínio em endereços IP,
permitindo que os utilizadores acedam a sites e serviços na Internet de forma mais fácil.

\subsection{Apache Webserver}
O Apache é um software de servidor HTTP de código aberto que permite a entrega de conteúdo web, como páginas HTML, imagens e vídeos,
para os utilizadores. É altamente configurável e extensível, permitindo a adição de módulos para suportar
diferentes funcionalidades, como autenticação, segurança e suporte a linguagens de programação.

\subsection{Virtual Hosts}
Os Virtual Hosts permitem que um único servidor Apache hospede múltiplos sites ou aplicações web, cada um com o seu próprio nome de domínio e configuração.
Isto é feito através da criação de entradas de configuração específicas para cada domínio.

%---------------------------------------------------------------------------------------------------------------------------
\newpage
\section{Configuração de Máquinas}\label{met}
A instalação AlmaLinux (\cite{almalinux}) , foi feita na máquina Servidor com o uso de um .iso fornecido pelo docente da UC.
Para a instalação foi necessário a criação de um disco de 7GB de espaço com três partições cujas quais são:
\begin{itemize}
	\item Partição /boot - 500MB em ext4
	\item Partição swap - 1000MB
	\item Partição / - Resto do Disco em ext4
\end{itemize}
Informações adicionais incluem 1MB de RAM e dois processadores de CPU e a rede está em modo 'bridge'.
No instalador foi definida a palavra-passe do 'root' como '123', KDUMP desativado e nomes de rede
'serverlab2.as.pt' para o Servidor.\\

\begin{figure}[h!]
	\centering
	\includegraphics[width=0.8\textwidth]{Recursos/UserMach.png}
	\caption{Exemplo de Maquina Utilizada}
	\label{fig:nfswarning}
\end{figure}

%---------------------------------------------------------------------------------------------------------------------------
\newpage
\section{Instalação de Pacotes}\label{pac}
A firewall foi desativada com o comando \textbf{systemctl disable --now firewalld}
e desativado o SELinux, necessitando a edição do ficheiro \textbf{/etc/selinux/config}
e alteração da linha \textbf{SELINUX=enforcing} para \textbf{SELINUX=disabled}.
Adicionalmente, foi necessário instalar os seguintes pacotes para a configuração dos serviços:\\
Servidor:
\begin{itemize}
	\item \textbf{nano} - Para a edição de ficheiros
	\item \textbf{whois} - Para verificar o IP da máquina
	\item \textbf{bind} - Para o DNS
	\item \textbf{bind-utils} - Para o DNS
	\item \textbf{httpd} - Para o Apache Webserver
\end{itemize}

%---------------------------------------------------------------------------------------------------------------------------
\newpage
\section{Configuração de Serviços}\label{conf}

\subsection{DNS}\label{dns}
Primeiramente é inicializado o DNS com o comando \textbf{systemctl enable --now named}.
Em seguida, para a configuração do DNS, foram feitas as seguintes alterações no ficheiro
\textbf{/etc/named.conf} com o uso do Putty:

\begin{itemize}
	\item Adicionado \textit{"any;"} nas linhas de \textit{listen-on port 53} e \textit{allow-query};
	\item Adicionadas as zonas forward e reverse.\\
\end{itemize}

\begin{minipage}{\textwidth}
	Zonas Forward
	\begin{verbatim}
    zone "pistas.gov" IN {
        type master;
        file "/var/named/pistas.gov.hosts";
        };


    zone "300emfrente.eu" IN {
        type master;
        file "/var/named/300emfrente.eu.hosts";
        };


    zone "then.com" IN {
        type master;
        file "/var/named/then.com.hosts";
    };
\end{verbatim}

\end{minipage}

\begin{minipage}{\textwidth}
	Zonas Reverse
	\begin{verbatim}
    zone "19.in-addr.arpa" IN {
    type master;
    file "/var/named/19.in-addr.arpa.hosts";
    };


    zone "3.in-addr.arpa" IN {
    type master;
    file "/var/named/3.in-addr.arpa.hosts";
    };


    zone "5.in-addr.arpa" IN {
    type master;
    file "/var/named/5.in-addr.arpa.hosts";
    };

    \end{verbatim}

\end{minipage}

%----------------------------------------------------------------------------------------------------------------------------
\newpage
Também necessitamos criar um ficheiro de configuração com caminho dado anteriormente em cada zona, cujos quais são:\\

\begin{minipage}{\textwidth}
	Zonas Foward
	\begin{verbatim}
        nano /var/named/pistas.gov.hosts
        $ttl 38400
        @ IN SOA serverlab2.as.pt. mail.as.com. (
        1165190726 ;serial
        10800 ;refresh
        3600 ; retry
        604800 ; expire
        38400 ; minimum
        )
            IN	NS	serverlab2.as.pt.
            IN	A	19.23.2.14
        ftp	IN	A	3.23.2.15
        webmail	IN	A	5.23.2.16
            IN	MX	10	as-smtp.300emfrente.eu.
    \end{verbatim}
\end{minipage}

\begin{minipage}{\textwidth}
	\begin{verbatim}
        nano /var/named/300emfrente.eu.hosts
        $ttl 38400
        @ IN SOA serverlab2.as.pt. mail.as.com. (
        1165190726 ;serial
        10800 ;refresh
        3600 ; retry
        604800 ; expire
        38400 ; minimum
        )
            IN	NS	serverlab2.as.pt.
            IN	A	114.21.1.14
        www	IN	A	177.8.90.1
        webmail	IN	A	31.21.1.16
        as-SMTP	IN	A	31.0.0.1
            IN	MX	10	as-smtp.300emfrente.eu.
    \end{verbatim}
\end{minipage}

\begin{minipage}{\textwidth}
	\begin{verbatim}
        nano /var/named/then.com.hosts
        $ttl 38400
        @ IN SOA serverlab2.as.pt. mail.as.com. (
        1165190726 ;serial
        10800 ;refresh
        3600 ; retry
        604800 ; expire
        38400 ; minimum
        )
            IN	NS	serverlab2.as.pt.
            IN	A	191.200.22.14
        ftp	IN	A	92.147.45.1
        webmail	IN	A	194.168.22.16
            IN	MX	10	as-smtp.300emfrente.eu.
    \end{verbatim}
\end{minipage}

\begin{minipage}{\textwidth}
	Zonas Reverse
	\begin{verbatim}
        nano /var/named/19.in-addr.arpa.hosts
        $ttl 38400
        @ IN SOA serverlab2.as.pt. mail.as.com. (
        1165192116
        10800
        3600
        604800
        38400 )
            IN	NS	serverlab2.as.pt.
        14.2.23	IN	PTR	pistas.gov.
    \end{verbatim}
\end{minipage}

\begin{minipage}{\textwidth}
	\begin{verbatim}
        nano /var/named/3.in-addr.arpa.hosts
        $ttl 38400
        @ IN SOA serverlab2.as.pt. mail.as.com. (
        1165192116
        10800
        3600
        604800
        38400 )
            IN	NS	serverlab2.as.pt.
        15.2.23	IN	PTR	ftp.pistas.gov.
    \end{verbatim}
\end{minipage}

\begin{minipage}{\textwidth}
	\begin{verbatim}
        nano /var/named/5.in-addr.arpa.hosts
        $ttl 38400
        @ IN SOA serverlab2.as.pt. mail.as.com. (
        1165192116
        10800
        3600
        604800
        38400 )
            IN	NS	serverlab2.as.pt.
        16.2.23	IN	PTR	webmail.pistas.gov.
    \end{verbatim}
\end{minipage}\\

Após serem configurados os ficheiros, é necessário reiniciar o
serviço DNS com o comando \textbf{systemctl restart named}.\\

\begin{figure}[h!]
	\centering
	\includegraphics[width=0.8\textwidth]{Recursos/DNS.png}
	\caption{Demonstração de funcionamento do DNS}
	\label{fig:nfswarning}
\end{figure}

Como podemos verificar, o DNS está a funcionar corretamente.
%---------------------------------------------------------------------------------------------------------------------------
\newpage
\subsection{Apache Webserver}\label{apache}
Primeiramente é inicializado o Apache com o comando \textbf{systemctl enable --now httpd}.
Também é necessária a criação de dois utilizadores \textbf{lab2user1} e \textbf{lab2user2} com as palavras-passe \textbf{lab2user1} e \textbf{lab2user2} respetivamente.\\
Estes utilizadores são adicionados ao grupo \textbf{users} com o uso dos comandos \textbf{usermod -aG users lab2user1} e \textbf{usermod -aG users lab2user2} respetivamente.
É criada a pasta \textbf{www} na diretoria \textbf{/home} de cada utilizador e para que exista uma página web, é criado o ficheiro \textbf{start.html} na diretoria criada.\\
Finalmente são dadas as permissões a cada utilizador com os comandos \textbf{chmod 755 /home/lab2user1 -R} e \textbf{chmod 755 /home/lab2user2 -R} respetivamente.\\

\begin{figure}[h!]
	\centering
	\includegraphics[width=0.8\textwidth]{Recursos/Apache1.png}
	\caption{Demonstração de funcionamento do Apache}
	\label{fig:nfswarning}
\end{figure}

Como podemos verificar, é possível aceder à pagina de cada utilizador através de um browser.\\

%---------------------------------------------------------------------------------------------------------------------------
\newpage
Para que exista uma pasta privada com autenticação, é criada a pasta \textbf{private} na diretoria \textbf{/home/www}.
Em seguida, é necessário criar os ficheiros \textbf{.htaccess} e \textbf{.filepasswd}, com os seguintes conteúdos respetivamente:\\\\
\begin{minipage}{\textwidth}
	\begin{verbatim}
        nano .htaccess
        AuthName "Diretorio Privado – lab2user1"
        AuthType Basic
        AuthUserFile /home/lab2user1/.filepasswd        
        require valid-user
    \end{verbatim}

	\begin{verbatim}
        nano .htaccess
        AuthName "Diretorio Privado – lab2user2"
        AuthType Basic
        AuthUserFile /home/lab2user2/.filepasswd        
        require valid-user
    \end{verbatim}

	\begin{verbatim}
        nano .filepasswd
        htpasswd -c /home/lab2user1/.filepasswd 123
        (Inserir palavra-passe '123')
    \end{verbatim}

	\begin{verbatim}
        nano .filepasswd
        htpasswd -c /home/lab2user2/.filepasswd fsp
        (Inserir palavra-passe 'fsp')
    \end{verbatim}
\end{minipage}

\newpage
\begin{figure}[h!]
	\centering
	\includegraphics[width=0.8\textwidth]{Recursos/Apache2.png}
	\caption{Demonstração da autenticação}
	\label{fig:nfswarning}
\end{figure}

\begin{figure}[h!]
	\centering
	\includegraphics[width=0.8\textwidth]{Recursos/Apache3.png}
	\caption{Demonstração da pasta privada}
	\label{fig:nfswarning}
\end{figure}

Como podemos verificar, as permissões estão a funcionar corretamente, sendo necessário inserir a palavra-passe para aceder à pasta privada.\\

%---------------------------------------------------------------------------------------------------------------------------
\newpage
\subsection{Virtual Hosts}\label{vhosts}
Primeiramente é necessária a criação de três utilizadores \textbf{low.org}, \textbf{circle3.pt} e \textbf{festas.pt} com a palavra-passe \textbf{123}.
É em seguida, criado um ficheiro \textbf{start.html} nas diretoria \textbf{/home} de cada utilizador.
Finalmente são dadas as permissões a cada utilizador com os comandos \textbf{chmod 755 /home/low.org -R}, \textbf{chmod 755 /home/circle3.pt -R} e \textbf{chmod 755 /home/festas.pt -R} respetivamente.\\

No ficheiro \textbf{/etc/named.conf} é necessário adicionar as seguintes linhas:\\\\
\begin{minipage}{\textwidth}
	\begin{verbatim}
        zone "low.org" IN {
            type master;
            file "/var/named/low.org.hosts";
        };

        zone "circle3.pt" IN {
            type master;
            file "/var/named/circle3.pt.hosts";
        };

        zone "festas.pt" IN {
            type master;
            file "/var/named/festas.pt.hosts";
        };
    \end{verbatim}
\end{minipage}

%---------------------------------------------------------------------------------------------------------------------------
\newpage
Também necessitamos criar um ficheiro de configuração com caminho dado anteriormente em cada zona, cujos quais são:\\

\begin{minipage}{\textwidth}
	\begin{verbatim}
        nano /var/named/low.org.hosts
        $ttl 38400
        @ IN SOA serverlab2.as.pt. mail.as.com. (
        1165190726 ;serial
        10800 ;refresh
        3600 ; retry
        604800 ; expire
        38400 ; minimum
        )
	        IN	NS	serverlab2.as.pt.
	        IN	A	192.168.77.39
        www	IN	A	192.168.77.39
    \end{verbatim}
\end{minipage}

\begin{minipage}{\textwidth}
	\begin{verbatim}
        nano /var/named/circle3.pt.hosts
        $ttl 38400
        @ IN SOA serverlab2.as.pt. mail.as.com. (
        1165190726 ;serial
        10800 ;refresh
        3600 ; retry
        604800 ; expire
        38400 ; minimum
        )
	        IN	NS	serverlab2.as.pt.
	        IN	A	192.168.77.39
        www	IN	A	192.168.77.39
    \end{verbatim}
\end{minipage}

\begin{minipage}{\textwidth}
	\begin{verbatim}
        nano /var/named/festas.pt.hosts
        $ttl 38400
        @ IN SOA serverlab2.as.pt. mail.as.com. (
        1165190726 ;serial
        10800 ;refresh
        3600 ; retry
        604800 ; expire
        38400 ; minimum
        )
	        IN	NS	serverlab2.as.pt.
	        IN	A	192.168.77.39
        www	IN	A	192.168.77.39
    \end{verbatim}
\end{minipage}

%----------------------------------------------------------------------------------------------------------------------------
Finalmente é necessário criar o ficheiro \textbf{/etc/httpd/conf/httpd.conf} com o seguinte conteúdo:\\\\
\begin{minipage}{\textwidth}
	\begin{verbatim}
        NameVirtualHost 192.168.77.39:9999
        <VirtualHost 192.168.77.39:9999>
        DocumentRoot "/home/low.org/"
        ServerName www.low.org
        ServerAlias low.org
        <Directory "/home/low.org">
        Options Indexes FollowSymLinks
        AllowOverride All
        Order allow,deny
        Allow from all
        Require method GET POST OPTIONS
        </Directory>
        </VirtualHost>
    \end{verbatim}
\end{minipage}

\begin{minipage}{\textwidth}
	\begin{verbatim}
        NameVirtualHost 192.168.77.39:7777
        <VirtualHost 192.168.77.39:7777>
        DocumentRoot "/home/low.org/"
        ServerName www.low.org
        ServerAlias low.org
        <Directory "/home/low.org">
        Options Indexes FollowSymLinks
        AllowOverride All
        Order allow,deny
        Allow from all
        Require method GET POST OPTIONS
        </Directory>
        </VirtualHost>
    \end{verbatim}
\end{minipage}

\begin{minipage}{\textwidth}
	\begin{verbatim}
        NameVirtualHost 192.168.77.39:9999
        <VirtualHost 192.168.77.39:9999>
        DocumentRoot "/home/circle3.pt/"
        ServerName www.circle3.pt
        ServerAlias circle3.pt
        <Directory "/home/circle3.pt">
        Options Indexes FollowSymLinks
        AllowOverride All
        Order allow,deny
        Allow from all
        Require method GET POST OPTIONS
        </Directory>
        </VirtualHost>
    \end{verbatim}
\end{minipage}

\begin{minipage}{\textwidth}
	\begin{verbatim}
        NameVirtualHost 192.168.77.39:7777
        <VirtualHost 192.168.77.39:7777>
        DocumentRoot "/home/circle3.pt/"
        ServerName www.circle3.pt
        ServerAlias circle3.pt
        <Directory "/home/circle3.pt">
        Options Indexes FollowSymLinks
        AllowOverride All
        Order allow,deny
        Allow from all
        Require method GET POST OPTIONS
        </Directory>
        </VirtualHost>
    \end{verbatim}
\end{minipage}

\begin{minipage}{\textwidth}
	\begin{verbatim}
        NameVirtualHost 192.168.77.39:9999
        <VirtualHost 192.168.77.39:9999>
        DocumentRoot "/home/festas.pt/"
        ServerName www.festas.pt
        ServerAlias festas.pt
        <Directory "/home/festas.pt">
        Options Indexes FollowSymLinks
        AllowOverride All
        Order allow,deny
        Allow from all
        Require method GET POST OPTIONS
        </Directory>
        </VirtualHost>
    \end{verbatim}
\end{minipage}

\begin{minipage}{\textwidth}
	\begin{verbatim}
        NameVirtualHost 192.168.77.39:7777
        <VirtualHost 192.168.77.39:7777>
        DocumentRoot "/home/festas.pt/"
        ServerName www.festas.pt
        ServerAlias festas.pt
        <Directory "/home/festas.pt">
        Options Indexes FollowSymLinks
        AllowOverride All
        Order allow,deny
        Allow from all
        Require method GET POST OPTIONS
        </Directory>
        </VirtualHost>
    \end{verbatim}
\end{minipage}

%---------------------------------------------------------------------------------------------------------------------------
\begin{figure}[h!]
	\centering
	\includegraphics[width=0.8\textwidth]{Recursos/VH.png}
	\caption{Demonstração dos Virtual Hosts}
	\label{fig:nfswarning}
\end{figure}

Como podemos verificar, após realizar os comandos \textbf{systemctl restart named} e \textbf{systemctl restart httpd}
os sites já se encontram funcionais.

%---------------------------------------------------------------------------------------------------------------------------
\newpage
\section{Conclusão}\label{con}
Todos os serviços foram implementados com sucesso, mesmo que o processo seja algo confuso face aos serviços implementados no Laboratório 1.
Este laboratório permitiu adquirir conhecimentos sobre Administração de Sistemas e as diferentes possibilidades de serviços que podem ser implementados.
Além disso, a configuração de Virtual Hosts e o uso do Apache Webserver são fundamentais para a criação de ambientes web complexos e escaláveis.
%---------------------------------------------------------------------------------------------------------------------------

\newpage
\renewcommand{\refname}{Bibliografia} % Para artigos
\renewcommand{\bibname}{Bibliografia} % Para livros e relatórios
\addcontentsline{toc}{section}{Bibliografia} % Adiciona a Bibliografia ao índice
\printbibliography
\newpage
\end{document}
