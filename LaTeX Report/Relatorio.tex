\documentclass[a4paper]{article}
% Pacotes necessários
\usepackage[portuguese]{babel}
\usepackage[backend=biber, style=apa, citestyle=apa, language=portuguese]{biblatex}
\usepackage{csquotes}
\addbibresource{Recursos/referencias.bib}

\usepackage{amsmath}
\usepackage{graphicx}
\usepackage{subcaption}
\usepackage{setspace}
\usepackage{siunitx} % Required for alignment
\sisetup{
  round-mode          = places, % Rounds numbers
  round-precision     = 2, % to 2 places
}
\usepackage{enumerate}
\usepackage{enumitem}
\usepackage{amsmath}
\usepackage{karnaugh-map}
\usepackage[section]{placeins}
\usepackage{geometry}
\usepackage{amssymb}
\usepackage{titling}
\usepackage[T1]{fontenc}
\usepackage{float}
\usepackage[hidelinks]{hyperref}
\usepackage{xcolor}
\usepackage{indentfirst}
\usepackage{array}
\usepackage{wrapfig} % Coloca isto no preâmbulo
\usepackage{soul}
\usepackage{afterpage}
\usepackage[toc,page]{appendix}
\newcolumntype{P}[1]{>{\centering\arraybackslash}p{#1}}
\onehalfspacing


% Comando para criar uma página vazia
\newcommand\myemptypage{
    \null
    \thispagestyle{empty}
    \addtocounter{page}{-1}
    \newpage
}

% Página de título principal
\newcommand{\firsttitlepage}{
    \begin{titlepage}
        \centering
        \vspace*{1cm}
        
        % Logos superior
        \begin{figure}[h!]
            \centering
            \includegraphics[width=6cm]{Recursos/Logos/LOGO_IPB} % Substitua pelo caminho da imagem
            \vspace{0.5cm}
        \end{figure}

        % Informações da instituição
        \large\textbf{INSTITUTO POLITÉCNICO DE BEJA} \\
        \large\textbf{Escola Superior de Tecnologia e Gestão} \\
        \large\textbf{Licenciatura em Engenharia Informática} \\
        \large\textbf{Tópicos de Engenharia Informática} \\
        
        \vspace{2cm}
        
        % Título do projeto
        {\Huge \textbf{WAF – Web Application Firewall}} \\
        
        \vspace{1.5cm}
        
        % Autores
    
        \large Martinho José Novo Caeiro - 23917 \\
        \large Paulo António Tavares Abade - 23919 \\
        
        \vfill
        
        % Logo inferior
        \begin{figure}[h!]
            \centering
            \includegraphics[width=6cm]{Recursos/Logos/IPBejaESTIG.jpg} % Substitua pelo caminho da imagem
        \end{figure}
        
        \vspace{0.5cm}
        
        % Local e data
        {\large Beja, março de 2025}
    \end{titlepage}
}

\newcommand{\secondtitlepage}{
    \begin{titlepage}
        \centering
        \vspace*{1cm}
        
        % Informações da instituição
        \large\textbf{INSTITUTO POLITÉCNICO DE BEJA} \\
        \large\textbf{Escola Superior de Tecnologia e Gestão} \\
        \large\textbf{Licenciatura em Engenharia Informática} \\
        \large\textbf{Tópicos de Engenharia Informática} \\
        
        \vspace{2cm}
        
        % Título do projeto
        {\Huge \textbf{WAF – Web Application Firewall}} \\
        
        \vspace{1.5cm}
        
        % Autores
        \large Martinho José Novo Caeiro - 23917 \\
        \large Paulo António Tavares Abade - 23919 \\

        \vspace{2cm}

        % Orientador
        \large Orientador: Professor Armando Ventura \\
        
        \vfill
        
        % Local e data
        {\large Beja, março de 2025}
    \end{titlepage}
}

\begin{document}


\pagenumbering{gobble} % Oculta numeração da página

% Primeira página de título
\firsttitlepage

\secondtitlepage


% Abstract
\section*{\LARGE\textbf{\textit{Resumo}}}

Neste relatório será abordado o processo de instalação da máquina
Servidor com o uso do AlmaLinux, os pacotes instalados e a configuração de
DNS, Virtual Hosts e WAF para a realização do Projeto.
Este relatório foi realizado no âmbito da Unidade Curricular de
Tópicos de Engenharia Informática (\cite{pagtei}).




\vspace{1cm}
% Keywords
\textbf{Keywords:} almalinux, putty, dns, virtual hosts, waf
\newpage
%--------------------------------------------------------------------------------------------------------------------------------------

\section*{\LARGE\textbf{\textit{Abstract}}}

In this report, we will address the installation process of the Server machine using
AlmaLinux, the installed packages and the configuration of DNS,
Virtual Hosts and WAF for the completion of the Project.


\vspace{1cm}
% Keywords
\textbf{Keywords:} almalinux, putty, dns, virtual hosts, waf
\renewcommand{\contentsname}{Índice}       % Título do sumário
\renewcommand{\listfigurename}{Índice de Figuras} % Título da lista de figuras

% Início do conteúdo do relatório
\newpage
\doublespacing
\tableofcontents
\listoffigures
\doublespacing

\newpage
\pagenumbering{arabic}

\section{Introdução}\label{intro}
Para a realização do projeto é necessária a configuração de duas máquina AlmaLinux, Servidor e WAF.
Em seguida são configurados os serviços DNS, Virtual Hosts e WAF.
Para a configuração de todos estes serviços foi utilizado o Putty (\cite{putty})
que permitiu inserir comandos mais rapidamente.
%---------------------------------------------------------------------------------------------------------------------------
\newpage
\section{Introdução Teórica}\label{obj}
\subsection{DNS}
O DNS (Domain Name System) é um sistema hierárquico e distribuído que permite a resolução de nomes de domínio em endereços IP,
permitindo que os utilizadores acedam a sites e serviços na Internet de forma mais fácil.

\subsection{Virtual Hosts}
Os Virtual Hosts permitem que um único servidor Apache hospede múltiplos sites ou aplicações web, cada um com o seu próprio nome de domínio e configuração.
Isto é feito através da criação de entradas de configuração específicas para cada domínio.

\subsection{WAF}
O WAF (Web Application Firewall) é uma solução de segurança que protege aplicações web contra ataques e vulnerabilidades.
Ele atua como uma camada adicional de defesa, analisando as requisições e respostas entre o cliente e o servidor, bloqueando atividades maliciosas e garantindo a integridade e confidencialidade dos dados.

%---------------------------------------------------------------------------------------------------------------------------
\newpage
\section{Configuração de Máquinas}\label{met}
A instalação AlmaLinux (\cite{almalinux}) , foi feita na máquina Servidor com o uso de um .iso fornecido pelo docente da UC.
Para a instalação foi necessário a criação de um disco de 8GB de espaço com três partições cujas quais são:
\begin{itemize}
	\item Partição /boot - 500MB em ext4
	\item Partição swap - 1000MB
	\item Partição / - Resto do Disco em ext4
\end{itemize}
Informações adicionais incluem 1MB de RAM e dois processadores de CPU e a rede está em modo 'bridge'.
No instalador foi definida a palavra-passe do 'root' como '1234', KDUMP desativado e nomes de rede
'server.tei.pt' e 'waf.tei.pt' para o Servidor e WAF respetivamente.\\

\begin{figure}[h!]
	\centering
	\includegraphics[width=0.8\textwidth]{Recursos/UserMach.png}
	\caption{Exemplo de Maquina Utilizada}
	\label{fig:nfswarning}
\end{figure}

%---------------------------------------------------------------------------------------------------------------------------
\newpage
\section{Instalação de Pacotes}\label{pac}
A firewall foi desativada com o comando \textbf{systemctl disable --now firewalld}
e desativado o SELinux, necessitando a edição do ficheiro \textbf{/etc/selinux/config}
e alteração da linha \textbf{SELINUX=enforcing} para \textbf{SELINUX=disabled}.
Adicionalmente, foi necessário instalar os seguintes pacotes para a configuração dos serviços:\\
Servidor:
\begin{itemize}
	\item \textbf{nano} - Para a edição de ficheiros
	\item \textbf{whois} - Para verificar o IP da máquina
	\item \textbf{bind} - Para o DNS
	\item \textbf{bind-utils} - Para o DNS
	\item \textbf{httpd} - Para as Virtual Hosts
	\item \textbf{mod\_ssl} - Para comunicações por SSL \\
\end{itemize}

WAF:
\begin{itemize}
	\item \textbf{nano} - Para a edição de ficheiros
	\item \textbf{httpd} - Para as Virtual Hosts
	\item \textbf{mod\_security} - Para o WAF
	\item \textbf{mod\_security\_crs} - Para o WAF
	\item \textbf{mod\_geoip} - Para o WAF
	\item \textbf{GeoIP} - Para o WAF
	\item \textbf{GeoIP\-data} - Para o WAF
\end{itemize}

%---------------------------------------------------------------------------------------------------------------------------
\newpage
\section{Configuração de Serviços}\label{conf}

\subsection{DNS}\label{dns}
Primeiramente é inicializado o DNS com o comando \textbf{systemctl enable --now named}.
Em seguida, para a configuração do DNS, foram feitas as seguintes alterações no ficheiro
\textbf{/etc/named.conf} com o uso do Putty:

\begin{itemize}
	\item Adicionado \textit{"any;"} nas linhas de \textit{listen-on port 53} e \textit{allow-query};
	\item Adicionadas as zonas forward e reverse.\\
\end{itemize}

\begin{minipage}{\textwidth}
	Zonas Forward
	\begin{verbatim}
    zone "trinta.org" {
        type master;
        file "/var/named/trinta.org.hosts";
    };

    zone "3emfrente.eu" {
        type master;
        file "/var/named/3emfrente.eu.hosts";
    };

    zone "the.com" {
        type master;
        file "/var/named/the.com.hosts";
    };
\end{verbatim}

\end{minipage}

%----------------------------------------------------------------------------------------------------------------------------
\newpage
Também necessitamos criar um ficheiro de configuração com caminho dado anteriormente em cada zona, cujos quais são:\\

\begin{minipage}{\textwidth}
	Zonas Foward
	\begin{verbatim}
        nano /var/named/trinta.org.hosts
        $TTL 38400
        @   IN  SOA server.tei.pt. admin.tei.pt. (
        1165190726 ; Serial
        10800      ; Refresh
        3600       ; Retry
        604800     ; Expire
        38400 )    ; Minimum TTL

	        IN	NS	server.tei.pt.
        server	IN	A	192.168.30.10
        @	IN	A	192.168.30.5
        www	IN	A	192.168.30.5
        www1	IN	A	192.168.30.10
    \end{verbatim}
\end{minipage}

\begin{minipage}{\textwidth}
	\begin{verbatim}
        nano /var/named/3emfrente.eu.hosts
        $TTL 38400
        @   IN  SOA server.tei.pt. admin.tei.pt. (
        1165190726 ; Serial
        10800      ; Refresh
        3600       ; Retry
        604800     ; Expire
        38400 )    ; Minimum TTL

	        IN	NS	server.tei.pt.
        server	IN	A	192.168.30.10
        @	IN	A	192.168.30.5
        www	IN	A	192.168.30.5
        www1	IN	A	192.168.30.10
    \end{verbatim}
\end{minipage}

\begin{minipage}{\textwidth}
	\begin{verbatim}
        nano /var/named/the.com.hosts
        $TTL 38400
        @   IN  SOA server.tei.pt. admin.tei.pt. (
        1165190726 ; Serial
        10800      ; Refresh
        3600       ; Retry
        604800     ; Expire
        38400 )    ; Minimum TTL

	        IN	NS	server.tei.pt.
        server	IN	A	192.168.30.10
        @	IN	A	192.168.30.5
        www	IN	A	192.168.30.5
        www1	IN	A	192.168.30.10
    \end{verbatim}
\end{minipage}\\

Após serem configurados os ficheiros, é necessário reiniciar o
serviço DNS com o comando \textbf{systemctl restart named}.\\

\begin{figure}[h!]
	\centering
	\includegraphics[width=0.8\textwidth]{Recursos/DNS.png}
	\caption{Demonstração de funcionamento do DNS}
	\label{fig:nfswarning}
\end{figure}

Como podemos verificar, o DNS está a funcionar corretamente.

%---------------------------------------------------------------------------------------------------------------------------
\newpage
\subsection{Virtual Hosts}\label{vhosts}
Primeiramente é necessária a criação de três utilizadores \textbf{trinta.org}, \textbf{3emfrente.pt} e \textbf{the.com} com a palavra-passe \textbf{123}.
É em seguida, é necessário criar o ficheiro \textbf{/etc/httpd/conf/httpd.conf} com o seguinte conteúdo:\\\\

\begin{minipage}{\textwidth}
	\begin{verbatim}
        NameVirtualHost 192.168.30.10:80
        <VirtualHost 192.168.30.10:80>
        DocumentRoot "/home/trinta.org/"
        ServerName www.trinta.org
        ServerAlias trinta.org
        <Directory "/home/trinta.org">
        Options Indexes FollowSymLinks
        AllowOverride All
        Order allow,deny
        Allow from all
        Require method GET POST OPTIONS
        </Directory>
        </VirtualHost>
    \end{verbatim}
\end{minipage}

\begin{minipage}{\textwidth}
	\begin{verbatim}
        NameVirtualHost 192.168.30.10:443
        <VirtualHost 192.168.30.10:443>
        DocumentRoot "/home/trinta.org/"
        ServerName www.trinta.org
        ServerAlias trinta.org
        <Directory "/home/trinta.org">
        Options Indexes FollowSymLinks
        AllowOverride All
        Order allow,deny
        Allow from all
        Require method GET POST OPTIONS
        </Directory>
        </VirtualHost>
    \end{verbatim}
\end{minipage}

\begin{minipage}{\textwidth}
	\begin{verbatim}
        NameVirtualHost 192.168.30.10:80
        <VirtualHost 192.168.30.10:80>
        DocumentRoot "/home/3emfrente.pt/"
        ServerName www.3emfrente.pt
        ServerAlias 3emfrente.pt
        <Directory "/home/3emfrente.pt">
        Options Indexes FollowSymLinks
        AllowOverride All
        Order allow,deny
        Allow from all
        Require method GET POST OPTIONS
        </Directory>
        </VirtualHost>
    \end{verbatim}
\end{minipage}

\begin{minipage}{\textwidth}
	\begin{verbatim}
        NameVirtualHost 192.168.30.10:443
        <VirtualHost 192.168.30.10:443>
        DocumentRoot "/home/3emfrente.pt/"
        ServerName www.3emfrente.pt
        ServerAlias 3emfrente.pt
        <Directory "/home/3emfrente.pt">
        Options Indexes FollowSymLinks
        AllowOverride All
        Order allow,deny
        Allow from all
        Require method GET POST OPTIONS
        </Directory>
        </VirtualHost>
    \end{verbatim}
\end{minipage}

\begin{minipage}{\textwidth}
	\begin{verbatim}
        NameVirtualHost 192.168.30.10:80
        <VirtualHost 192.168.30.10:80>
        DocumentRoot "/home/the.com/"
        ServerName www.the.com
        ServerAlias the.com
        <Directory "/home/the.com">
        Options Indexes FollowSymLinks
        AllowOverride All
        Order allow,deny
        Allow from all
        Require method GET POST OPTIONS
        </Directory>
        </VirtualHost>
    \end{verbatim}
\end{minipage}

\begin{minipage}{\textwidth}
	\begin{verbatim}
        NameVirtualHost 192.168.30.10:443
        <VirtualHost 192.168.30.10:443>
        DocumentRoot "/home/the.com/"
        ServerName www.the.com
        ServerAlias the.com
        <Directory "/home/the.com">
        Options Indexes FollowSymLinks
        AllowOverride All
        Order allow,deny
        Allow from all
        Require method GET POST OPTIONS
        </Directory>
        </VirtualHost>
    \end{verbatim}
\end{minipage}

%---------------------------------------------------------------------------------------------------------------------------
\begin{figure}[h!]
	\centering
	\includegraphics[width=0.8\textwidth]{Recursos/VH.png}
	\caption{Demonstração dos Virtual Hosts}
	\label{fig:nfswarning}
\end{figure}

Como podemos verificar, após realizar os comandos \textbf{systemctl restart named} e \textbf{systemctl restart httpd}
os sites já se encontram funcionais.

%---------------------------------------------------------------------------------------------------------------------------
\newpage
\section{WAF}\label{con}

%---------------------------------------------------------------------------------------------------------------------------
\newpage
\section{Conclusão}\label{con}
Todos os serviços foram implementados com sucesso, dado que a sua maioria foi semelhante ao Laboratório 2 da Unidade Curricular de Administração de Sistemas.
Este projeto permitiu adquirir conhecimentos sobre este Tópico e as diferentes possibilidades de manter os serviços seguros.
Além disso, a configuração de WAF é essencial para garantir a segurança das aplicações web, protegendo-as contra ataques e vulnerabilidades.
%---------------------------------------------------------------------------------------------------------------------------

\newpage
\renewcommand{\refname}{Bibliografia} % Para artigos
\renewcommand{\bibname}{Bibliografia} % Para livros e relatórios
\addcontentsline{toc}{section}{Bibliografia} % Adiciona a Bibliografia ao índice
\printbibliography
\newpage
\end{document}
